%! Author = Martin Vandenbussche

In this chapter, we will provide an overview of what has been done in previous works regarding NewOz, which will serve as a starting point for our reflexions.\newline
Last year's work of Jean-Pacifique Mbonyincungu used a very systematic approach consisting of reviewing a lot of languages features and syntax elements of Oz.
For each of these, code snippets in both Oz and Scala/Ozma\footnote{TODO describe briefly} were provided and compared.
The code served as a basis for the following discussion, comparing pros and cons both syntax's, and motivating the final choices being made.
The process was rationalized by using a set of objective factors, allowing to rate each choice on a numeric scale in an attempt to provide the best syntax for each language feature.
\newline
The two main results of this work were the definition of a new syntax, described in~\ref{sec:appendix-a} as an EBNF grammar, as well as the writing of a Parser, which is able to convert any code written in NewOz to the equivalent Oz code.

\section{Current state of the NewOz grammar}\label{sec:ch1CurrentGrammar}
Talk about pros and cons of Jean-Pacifique's NewOz

\section{The NewOz Parser}\label{sec:ch1CurrentParser}
Talk about pros and cons of Jean-Pacifique's Scala Parser

\section{Other works}\label{sec:ch1OtherWorks}
Briefly present Ozma + other relevant works