\documentclass[a4paper,11pt]{scrreprt}
%! Author = Martin Vandenbussche

%\usepackage[utf8]{inputenc} % inputenc package ignored with utf8 based engines.
\usepackage[T1]{fontenc}

\usepackage{amsmath,amsthm,amssymb}

\usepackage{listings}
\usepackage{graphicx}
\usepackage{url}

\usepackage{comment}

\usepackage[backend=biber,style=numeric,citestyle=numeric]{biblatex}
\usepackage{csquotes}
\addbibresource{biblio.bib}

\usepackage{xcolor}
\definecolor{red}{RGB}{255,0,0}

\usepackage[toc,page]{appendix}

\title{Master Thesis - A new syntax for the Oz programming language}
\subject{Under the supervision of Prof. Peter Van Roy}
\author{Martin Vandenbussche - 02441500}
\date{Academic year 2020--2021}

\begin{document}

\maketitle

\tableofcontents

%! Author = Martin Vandenbussche

\paragraph{Abstract}
\textit{
The Oz programming language has proven over the years its value as a learning and research tool \textcolor{red}{about} programming paradigms, in many universities around the world.
It has had a major influence over different more recent programming languages, and has functionally stood the test of time.
That being said, its syntax lacks the ability the efficiently use some modern programming paradigms;
the goal of this work, building upon last year's thesis of Jean-Pacifique Mbonyincungu, is to design a brand new syntax for Oz, that will allow the language to tackle new paradigms, while still being compatible with the existing Mozart system.
}
\medskip
\paragraph{Technical notes}
One of the key elements of this project is that compatibility has to be maintained with the existing Mozart system, for the official release of Mozart2.
The idea of writing a new compiler has thus quickly been set aside, as it would drastically increase the time and complexity requirements of the project.
Instead, the previous thesis brought forward the idea of writing a syntax parser, that would serve as a compatibility layer between the so-called "newOz" syntax, and the existing Oz syntax.\cite{jpthesis}
NewOz code will be translated, and then fed to the existing Oz compiler.\newline
This approach has been selected because it is easier to implement, relies on more modern technologies than the existing Mozart compiler, and will allow for more flexibility down the line because of the nature of the underlying Scala code of the Parser.
We should however keep in mind the major limitation that this approach brings : error messages generated by the Mozart compiler will be way harder to interpret by the end-user.
One of the goals of the project being that this Parser-to-compiler behavior stays transparent to the programmer, we should keep in mind that future Oz learners will know nothing of the current Oz syntax.
As such, the compiler output will probably be obscure to them, and it is probably a good idea to try and alleviate this confusion as much as possible.

\chapter{Previous works}\label{ch:1}
%! Author = Martin Vandenbussche

In this chapter, we will provide an overview of what has been done in previous works regarding NewOz, which will serve as a starting point for our reflexions.\newline
Last year's work of Jean-Pacifique Mbonyincungu used a very systematic approach consisting of reviewing a lot of languages features and syntax elements of Oz.
For each of these, code snippets in both Oz and Scala/Ozma\footnote{TODO describe briefly} were provided and compared.
The code served as a basis for the following discussion, comparing pros and cons both syntax's, and motivating the final choices being made.
The process was rationalized by using a set of objective factors, allowing to rate each choice on a numeric scale in an attempt to provide the best syntax for each language feature.
\newline
The two main results of this work were the definition of a new syntax, described in~\ref{sec:appendix-a} as an EBNF grammar, as well as the writing of a Parser, which is able to convert any code written in NewOz to the equivalent Oz code.

\section{Current state of the NewOz grammar}\label{sec:ch1CurrentGrammar}
Talk about pros and cons of Jean-Pacifique's NewOz

\section{The NewOz Parser}\label{sec:ch1CurrentParser}
Talk about pros and cons of Jean-Pacifique's Scala Parser

\section{Other works}\label{sec:ch1OtherWorks}
Briefly present Ozma + other relevant works

\begin{appendices}
\section*{Appendix A : newOz EBNF Grammar}\label{sec:appendix-a}
\begin{lstlisting}[label={lst:newOzEBNF},frame=single,basicstyle=\footnotesize,escapeinside={(*}{*)}]
EBNF grammar for newOz, suitable for recursive descent
- Note that the concatenation symbol in EBNF (comma) is
omitted for readability reasons
- <T> represents a generic definition
(think about Java generic types)
Notation      Meaning
===========================================================================
(*\epsilon*)             singleton containing the empty word
(*$(w)$*)           grouping of regular expressions
(*$[w]$*)           union of \epsilon with the set of words w (optional group)
(*$\{w\}$*)          zero or more times w
(*$\{w\}+$*)         one or more times w
(*$w_1~w_2$*)         concatenation of (*$w_1$*) with (*$w_2$*)
(*$w_1 | w_2$*)         logical union of (*$w_1$*) and (*$w_2$*) (OR)
(*$w_1-w_2$*)       difference of (*$w_1$*) and (*$w_2$*)

// Interactive statements
interStatement ::= statement
| DECLARE LCURLY {declarationPart}+ [interStatement] RCURLY

statement ::= nestCon(statement)
| nestDec(variable)
| SKIP
| statement statement

expression ::= nestCon(expression)
| nestDec(DOLLAR)
| expressions evalBinOp expression
| DOLLAR
| term
| monOp expression
| THIS
| UNDERSCORE //TODO to stay coherent with oldOz ?

//TODO expression simplifiable ?
inStatement ::= LCURLY [{declarationPart}+] statement RCURLY

//TODO expression simplifiable ?
inExpression ::= LCURLY [{declarationPart}+]
                [statement] expression RCURLY

in(statement) ::= inStatement

in(expression) ::= inExpression

// This is in the sense of the old Oz "in" keyword
blockIn(<T>) ::= LCURLY (in(<T>) | WHITESPACE) RCURLY
//TODO whitespace rly ou erreur ds these ?

//Here, <T> is statement or expression
nestCon(<T>) ::= expression (ASSIGN | DEFINE) expression
| expression LPAREN {expression COMMA} RPAREN
| blockIn(<T>)
| LPAREN in(<T>) RPAREN
| IF LPAREN expression RPAREN blockIn(<T>)
  {ELSE IF LPAREN expression RPAREN blockIn(<T>)}
  [ELSE blockIn(<T>)]
| MATCH expression LCURLY {CASE pattern
   [(LAND | LOR) expression] RBARROW (blockIn(<T>) | in(<T>))}+
  [ELSE blockIn(<T>)] RCURLY
| FOR LPAREN {loopDec}+ RPAREN blockIn(<T>)
| TRY blockIn(<T>) [CATCH LCURLY
   {CASE pattern IMPL (blockIn(<T>) | in(<T>))} RCURLY]
  [FINALLY blockIn(<T>)]
| RAISE blockIn(expression)
| THREAD blockIn(<T>)
| LOCK [LPAREN expression RPAREN] blockIn(<T>)

//Here, <T> is always DOLLAR or a variable
nestDec(T) ::= DEFPROC <T> LPAREN {pattern COMMA} RPAREN
  blockIn(statement)
| DEF [LAZY] <T> LPAREN {pattern COMMA} RPAREN
  blockIn(expression)
//TODO the following lines are a mess -> rewrite from thesis' text
| FUNCTOR [<T>] {
[IMPORT ({variable [AT atom]
| variable LPAREN {(atom | int) [COLON variable] COMMA}+ RPAREN) COMMA
}+]
[EXPORT {[(atom | int) COLON] variable COMMA}+]
}
blockIn(statement)
| CLASS <T> {classDescriptor} LCURLY
{DEF methHead [ASSIGN variable] (blockIn(expression) | blockIn(statement))} RCURLY

// Terms and patterns
term ::= [LNOT] variable | int | float | character | atom
| string | UNIT | TRUE | FALSE | UNDERSCORE //TODO UNDERSCORE : seems to me but not in the book ?!
| label LPAREN {[feature COLON] expression COMMA} RPAREN
| expression consBinOp expression
| LBRACK {expression COMMA}+ RBRACK

pattern ::= [LNOT] variable | int | float | character | atom
| string | UNIT | TRUE | FALSE | UNDERSCORE //TODO UNDERSCORE : seems to me but not in the book ?!
| label LPAREN {[feature COLON] pattern COMMA}
  [COMMA DOTDOTDOT] RPAREN
| pattern consBinOp pattern
| LBRACK {pattern}+ RBRACK

declarationPart ::= {(VAL | VAR) {(variable | pattern)
  ASSIGN (expression | statement) COMMA}+ [SEMI]}

loopDec ::= variable IN expression
  [DOTDOT expression] [SEMI expression]
| variable IN expression SEMI expression SEMI expression
| break COLON variable
| CONTINUE COLON variable
| RETURN COLON variable
| DEFAULT COLON expression
| COLLECT COLON variable

binaryOp ::= evalBinOp | consBinOp

consBinOp ::= HASHTAG | PIPE

evalBinOp ::= PLUS | MINUS | STAR | SLASH | MODULO | DOT
| LAND | LOR | DEFINE | COMMA | ASSIGN | EQUAL
| NE | LT | LE | GT | GE

label ::= UNIT | TRUE | FALSE | variable | atom

feature ::= UNIT | TRUE | FALSE | variable | atom | int

classDescription ::= EXTENDS {expression}+
| PROP {expression}+
| ATTR {attrInit}+

attrInit ::= ([LNOT] variable | atom | UNIT | TRUE | FALSE)
  [COLON expression]

methHead ::= ([LNOT] varStrict | atomLisp | UNIT | TRUE | FALSE)
  [LPAREN {methArg COMMA} [COMMA DOTDOTDOT] RPAREN]
  [ASSIGN variable]

methArg ::= [feature COLON] (variable | UNDERSCORE | DOLLAR)
  [ASSIGN expression]

varStrict ::= UPPERCASE {ALPHANUM}
| APOSTROPHE {VARIABLECHAR | PSEUDOCHAR} APOSTROPHE

variable ::= (UPPERCASE | LOWERCASE) {ALPHANUM}
| APOSTROPHE {VARIABLECHAR | PSEUDOCHAR} APOSTROPHE

atom ::= atomLisp
| RACCENT {ATOMCHAR | PSEUDOCHAR} RACCENT

string ::= QUOTE {STRINGCHAR | PSEUDOCHAR} QUOTE

character ::= CHARINT
| AND CHARCHAR
| AND PSEUDOCHAR

atomLisp ::= APOSTROPHE (LOWERCASE | UPPERCASE)
  {alphanumericChar} - keywords //TODO

\end{lstlisting}


\section*{Appendix B : lexical grammar}\label{sec:appendix-b}
\begin{lstlisting}[label={lst:newOzLexical},frame=single,basicstyle=\footnotesize\ttfamily,escapeinside={(*}{*)}]
Lexical grammar for newOz
Notation      Meaning
===========================================================================
(*\epsilon*)             singleton containing the empty word
(*$(w)$*)           grouping of regular expressions
(*$[w]$*)           union of \epsilon with the set of words w (optional group)
(*$\{w\}$*)          zero or more times w
(*$\{w\}+$*)         one or more times w
(*$w_1~w_2$*)         concatenation of (*$w_1$*) with (*$w_2$*)
(*$w_1 | w_2$*)         logical union of (*$w_1$*) and (*$w_2$*) (OR)
(*$w_1-w_2$*)       difference of (*$w_1$*) and (*$w_2$*)

// White spaces - ignored
WHITESPACE ::= (" "|"\b"|"\t"|"\n"|"\r"|"\f")

// Comments - ignored
("//" {~("\n"|"\r")} ("\n"|"\r"["\n"])) | "?"

// Multi-line comments - ignored
"/*" {CHAR - "*/"} "*/"

// Reserved keywords
//ANDTHEN ::= "andthen"
AT      ::= "at"
ATTR    ::= "attr"
BREAK   ::= "break"
CASE    ::= "case"
CATCH   ::= "catch"
//CHOICE  ::= "choice"
CLASS   ::= "class"
//COLLECT ::= "collect"
//COND    ::= "cond"
CONTINUE  ::= "continue"
DECLARE ::= "declare"
DEF     ::= "def"
DEFPROC ::= "defproc"
DEFAULT ::= "default"
//DEFINE  ::= "define"
//DIS     ::= "dis"
//DIV     ::= "div"
DO      ::= "do"
ELSE    ::= "else"
//ELSECASE  ::= "elsecase"
//ELSEIF  ::= "elseif"
//ELSEOF  ::= "elseof"
//END     ::= "end"
EXPORT  ::= "export"
EXTENDS ::= "extends"
//FAIL    ::= "fail"
FALSE   ::= "false"
//FEAT    ::= "feat"
FINALLY ::= "finally"
FOR     ::= "for"
FROM    ::= "from"
//FUN     ::= "fun"
FUNCTOR ::= "functor"
IF      ::= "if"
IMPORT  ::= "import"
IN      ::= "in"
LAZY    ::= "lazy"
//LOCAL   ::= "local"
LOCK    ::= "lock"
MATCH   ::= "match"
METH    ::= "meth"
//MOD     ::= "mod"
NIL     ::= "nil"
//NOT     ::= "not"
//OF      ::= "of"
OR      ::= "or"
//ORELSE  ::= "orelse"
//PREPARE ::= "prepare"
//PROC    ::= "proc"
PROP    ::= "prop"
RAISE   ::= "raise"
//REQUIRE ::= "require"
RETURN  ::= "return"
//SELF    ::= "self"
SKIP    ::= "skip"
//THEN    ::= "then"
THIS    ::= "this"
THREAD  ::= "thread"
TRUE    ::= "true"
TRY     ::= "try"
UNIT    ::= "unit"
VAL     ::= "val"
VAR     ::= "var"

ASSIGN      ::= "=" ok
DEFINE      ::= ":=" ok
PLUSASS  ::= "+=" ok
MINUSASS ::= "-=" ok
EQUAL       ::= "==" ok
NE          ::= "\=" ok
LT          ::= "<" ok
GT          ::= ">" ok
LE          ::= "=<" ok
GE          ::= ">=" ok
LBARROW     ::= "<=" ok
IMPL        ::= "=>" ok
AND         ::= "&" ok TODO DELETED
LAND        ::= "&&" ok
PIPE        ::= "|" ok TODO DELETED
LOR         ::= "||" ok
LNOT        ::= "!" ok
LNOTNOT     ::= "!!" ok
MINUS       ::= "-" ok
PLUS        ::= "+" ok
STAR        ::= "*" ok
SLASH       ::= "/" ok
BACKSLASH   ::= "\" ok
MODULO      ::= "%" ok
HASHTAG     ::= "#" ok
UNDERSCORE  ::= "_" ok
DOLLAR      ::= "$" ok
APOSTROPHE  ::= "'" ok
QUOTE       ::= """ ok
LACCENT     ::= "`" ok
RACCENT     ::= "´" ok
HAT         ::= "^" ok
BOX         ::= "[]" ok
//TILDE       ::= "~" ok
DEGREE      ::= "°" ok
//COMMERCAT   ::= "@" ok
//LARROW      ::= "<-" ok
//RARROW      ::= "->" ok
//FDASSIGN    ::= "=:" //skipped
//FDNE        ::= "\=:" //skipped
//FDLT        ::= "<:" //skipped
//FDLE        ::= "=<:" //skipped
//FDGT        ::= ">:" //skipped
//FDGE        ::= ">=:" //skipped
COLCOL      ::= "::" ok
//COLCOLCOL   ::= ":::" ok

COMMA       ::= "," ok
DOT         ::= "." ok
LBRACK      ::= "[" ok
LCURLY      ::= "{" ok
LPAREN      ::= "(" ok
RBRACK      ::= "]" ok
RCURLY      ::= "}" ok
RPAREN      ::= ")" ok
SEMI        ::= ";" ok
COLON       ::= ":" ok
DOTDOT      ::= ".." ok
ELLIPSIS    ::= "..." ok

// Literals
UPPERCASE      ::= "A"|...|"Z" ok
LOWERCASE      ::= "a"|...|"z" ok
DIGIT          ::= "0"|...|"9" ok
NONZERODIGIT   ::= "1"|...|"9" ok
CHARINT        ::= "0"|...|"255" ok
ALPHANUM       ::= UPPERCASE | LOWERCASE | DIGIT | "_" ok
ATOMCHAR       ::= CHAR - ("'"|"\")
STRINGCHAR     ::= CHAR - ("""|"\")
VARIABLECHAR   ::= CHAR - ("`"|"\")
CHARCHAR       ::= CHAR - ("\")
ESCCHAR        ::= "a"|"b"|"f"|"n"|"r"|"t"|"v"|"\"|"'"|"""|"`"|"°"
OCTDIGIT       ::= "0"|...|"7" ok
HEXDIGIT       ::= "0"|...|"9"|"A"|...|"F"|"a"|...|"f" ok
BINDIGIT       ::= "0"|"1" ok
NONZERODIGIT   ::= "1"|...|"9" ok
PSEUDOCHAR     ::= "\" OCTDIGIT OCTDIGIT OCTDIGIT ok
                | "\" ("x" | "X") HEXDIGIT HEXDIGIT ok

// End of file
EOF            ::= "<end of file>"

\end{lstlisting}
\end{appendices}
\begin{comment}
Promoters:
Peter Van Roy
Other people :
Laurent Nicolas (INGI)
Description:
Oz is a well-factored language that supports many programming paradigms and is successfully being used in courses and MOOCs.
It has also influenced the development of other languages such as Scala and Go.
The internal structure of Oz has stood the test of time.
But the syntax has not: there are many idioms used today that Oz syntax supports poorly.
For example, big-data style computations often use a composition of functional objects, with a syntax something like DB.select_list(f1).map(f2).sort(f3).reduce(f4,u), with a chain of operations such that each operation takes a functional object and returns a new functional object.
Oz syntax does not support this well.

This master's project continues the work of an earlier master's project on improving Oz syntax.
The ultimate goal is to change the Oz syntax in the official release of Mozart 2, which is a major result.
That is why this work is spread over several years, in several master's projects.
The goal is to make a completely new syntax of Oz, taking advantage of the best parts of Scala, Python, Go, Clojure, and other modern languages.
The new syntax should be simple, well-factored, and support sophisticated programming idioms, while still keeping the clean Oz kernel language semantics.
You will study important programming idioms in many languages and design and implement a new parser front end for the Mozart 2 system.

References
[1] Peter Van Roy and Seif Haridi.  Concepts, Techniques, and Models of Computer Programming.  MIT Press, 2004.
[2] Peter Van Roy, Seif Haridi, Christian Schulte, and Gert Smolka.  A History of the Oz Multiparadigm Language.  The Fourth ACM SIGPLAN History of Programming Languages Conference, 2020.
[3] The Mozart Programming System, http://www.mozart2.org
[4] Welcome to Python.org, http://www.python.org
[5] The Scala Programming Language, http://www.scala-lang.org
[6] Erlang Programming Language, https://www.erlang.org
[7] The Go Programming Language, https://golang.org
[8] Clojure, https://clojure.org
[9] François Fonteyn, Comprehensions in Mozart, Master's thesis, ICTEAM Institute, Université catholique de Louvain, June 2014.
\end{comment}

\printbibliography

\end{document}